\documentclass[letterpaper,yyyy,draft]{simpleresumecv}

\newcommand{\CVAuthor}{Sergio A. Ulloa B.}
\newcommand{\CVWebpage}{https://sulloa.com}

\hypersetup{
pdfauthor={\CVAuthor},
pdfsubject={\CVWebpage},
pdfcreator={XeLaTeX},
pdfproducer={},
pdfkeywords={Hello there, yeah I add keywords to my PDFs, programmer, programming, dev, developer, C++, golang, go, python, python3, linux, docker, latex, freelancer, freelance, dev, developer, kicad, electronics design, iot, internet of things, junior, senior, semi-senior, semi senior, svelte, sveltekit, svelte kit, ts, typescript, arduino, micropython, embedded development, cad, fusion360},
unicode=true,
pdfstartview=FitH,
pdfpagelayout=OneColumn,
hidelinks,
breaklinks}

\newcommand{\Code}[1]{\mbox{\textbf{#1}}}
\newcommand{\CodeCommand}[1]{\mbox{\textbf{\textbackslash{#1}}}}

\begin{document}

\makeatletter
\let\@oddfoot\@empty{}
\let\@evenfoot\@empty{}
\makeatother

\Title{\CVAuthor}

{
    \begin{SubTitle}
    \href{mailto:sergioulloa@protonmail.com}{sergioulloa@protonmail.com}
    \,\SubBulletSymbol\,
    \href{\CVWebpage}{\url{\CVWebpage}}
    \end{SubTitle}
}

\begin{Body}

% WORK EXPERIENCE

\Section
{Experiencia\newline
Laboral}
{Work Experience}
{PDF:ExperienciaLaboral}
{
    \Entry{
        \href{https://happyvolt.com/}{\textbf{HappyVolt}},
        La Reina, RM, Chile

        \BulletItem{
            \textbf{Consultor Técnico}
        }
        \hfill
        Agosto 2020 -- Diciembre 2020

        \BulletItem{
            \textbf{Consultor Técnico}
        }
        \hfill
        Noviembre 2019 -- Febrero 2020

        \BulletItem{
            \textbf{Practica Laboral}
        }
        \hfill
        Febrero 2019 -- Noviembre 2019

        \BulletItem{
            \textbf{Roles}
        }
        \hfill
        \begin{Detail}
            \SubBulletItem{
                Técnico Electrónico
            }

            \SubBulletItem{
                Diseñador Electrónico
            }

            \SubBulletItem{
                Ingeniero de Software
            }
        \end{Detail}

        \BulletItem{
            \textbf{Logros}
        }

        \begin{Detail}
        \SubBulletItem{
            Dirigí el proceso de diseño de hardware para un robot dedicado al análisis de fruta orientado a clientes de la industria agrícola.
        }
        \end{Detail}
    }
}

% EDUCATION

\Section
{Educación}
{Educación}
{PDF:Educacion}
{
    \Entry{
        \href{https://www.uautonoma.cl/}{\textbf{Universidad Autónoma de Chile}},
        RM, Chile
        \hfill
        \DatestampY{2021} -- \DatestampY{2022}

        \BulletItem{
            \textbf{Ingeniería Civil Informática}
        }

        \SubBulletItem{
            4 Semestres cursados
        }
    }

    \Entry{
        \href{https://www.lichan.cl/}{\textbf{Liceo Industrial Bicentenario Chileno Alemán}},
        RM, Chile
        \hfill
        \DatestampY{2015} -- \DatestampY{2019}

        \BulletItem{
            \textbf{Titulado en Técnico de Electrónico}
        }

        \BulletItem{
            \textbf{Logros}
        }

        \begin{Detail}
        \SubBulletItem{
            Representante en \href{https://worldskills.org/}{\textbf{WorldSkills Chile}} 2017 y \href{https://worldskills.org/}{\textbf{WorldSkills Chile}} 2018, donde obtuve el segundo lugar en electrónica y segundo lugar en robótica, respectivamente.
        }
        \end{Detail}
    }
}

% SKILLS

\Section
{Habilidades Técnicas}
{Habilidades Técnicas}
{PDF:HabilidadesTecnicas}
{
    \Entry{
        \textbf{Lenguajes De Programación}

        \begin{Detail}
        \BulletItem{
            Golang, Python, Arduino, C++, Bash
        }
        \end{Detail}
    }

    \Entry{
        \textbf{Desarrollo Web}

        \begin{Detail}
        \BulletItem{
            SvelteKit, TailwindCSS, TypeScript
        }
        \end{Detail}
    }

    \Entry{
        \textbf{Diseño CAD y Electrónico}

        \begin{Detail}
        \BulletItem{
            KiCad, Eagle, Fusion360, Desarrollo Integrado
        }
        \end{Detail}
    }

    \Entry{
        \textbf{Entorno de Desarrollo Y Herramientas de Productividad}

        \begin{Detail}
        \BulletItem{
            Linux, Git, Docker, NeoVim, VS Code, Latex, Suite de Microsoft Office, RegExp
        }
        \end{Detail}
    }
}

%\newpage

% ACHIEVEMENTS

\Section
{Logros}
{Logros}
{PDF:Logros}
{
    \Entry{
        \textbf{Capture The Flag}
        \BulletItem{
            \href{https://8dot8.org}{\textbf{8dot8 CTF}}
        }
        \hfill

        \begin{Detail}
        \SubBulletItem{
            2022 -- Tercer Lugar -- Solución de desafíos de Cyphers, Ingeniería Inversa y Criptografía.
        }
        \end{Detail}
    }

    \Entry{
        \textbf{Hackatones}
        \Entry{
            \BulletItem{
                \href{https://angelhack.com}{\textbf{AngelHack Global Hackathon Series}}
            }
            \hfill

            \begin{Detail}
            \SubBulletItem{
                2019 -- Ganador AngelHack Santiago y EY Fintech challenge -- \newline Desarrollo de un banco digital con focus en nuevos trabajadores e inmigrantes.
            }
            \SubBulletItem{
                2018 -- Ganador Hurify Open Challenge -- Desarrollo de un rastreador BLE.
            }
            \end{Detail}
        }
        \Entry{
            \BulletItem{
                \href{https://spaceappschallenge.org}{\textbf{Nasa SpaceApps Challenge}}
            }

            \begin{Detail}
            \SubBulletItem{
                2017 -- Desarrollo de IgniSight, una red de nodos LoRa capaz\newline de detectar incendios forestales.
            }
            \SubBulletItem{
                2016 -- Primer Lugar, Tecnología -- Desarrollo de un cohete modular.
            }
            \end{Detail}
        }
    }
}

% LANGUAGES

\Section
{Idiomas}
{Idiomas}
{PDF:Idiomas}
{
    \Entry{
        \BulletItem{
            Español: Lenguaje materno
        }
    }

    \Entry{
        \BulletItem{
            Inglés: Fluido (habla, lectura, escritura)
        }

        \SubBulletItem{
            Certificación TOEFL Junior B2
        }
    }
}
\end{Body}

\end{document}
